% !Mode:: "TeX:UTF-8"

\cabstract{

随着社会发展,有更多的工件材料被生产出来,对于表面的质量要求也逐渐增高,表面的质量问题也逐渐被人们所关注,但是在工件生产过程中,收到生产工艺或者机器影响,工件上经常会出现凹陷、凸起、划痕等表面缺陷问题。因此逐步形成了表面缺陷检测技术。表面缺陷检测技术是自动化工业生产中的重要一环,在整个生产环节中,产生表面缺陷,形成部件缺失或者是部件错位,除了检测精度以外,在资源有限的自动表面检测解决方案上也需要在易用性和处理时间方面也同时具有优异的性能。本文提出了一种基于CornerNet的神经网络结构,用于检测一些常见的表面缺陷。在数据集的基础上,基于图像的降噪、预处理和数据增强,增加数据的可用性。并使用多层的fire module用以优化网络,并使用Hourglass模块构建整个网络结构,通过深层可分离卷积用以降低整个网络的运算需求,也提高了整个模型的检测精度。实验表明,与主流的卷积神经网络相比,我们的模型精确性和实时性都有保证,能够满足表面缺陷检测的要求。
}

\ckeywords{表面缺陷;机器视觉;深度学习;图像处理;神经网络}

\eabstract{
    With the development of morden society, more workpiece are produced, and the surface quality requirements are increasing. To stand test of time, the quality of surface is gradually concerned by people. However, After production process, workpiece may take damage due to the production process and influence of machine, Such as depressions, bumps, and scratches often occur on the workpiece. Therefore, surface defect detection technology has gradually formed and become an important part of automated industrial production. In the production process, missing parts or parts are dislocated may be the surface defects. Despite detection accuracy, it is also required in automatic surface inspection solutions with limited resources. It also need to have excellent performance in terms of ease of use as well as that in processing time. This paper presents a neural network structure based on CornerNet for detecting some common surface defects. Based on the dagm dataset, noise reduction, pre-processing and image augmentation are applyed to incerase data credibility. The entire network structure is built with the hourglass module, and multi-layer fire modules are included to optimize our network. The deep separable convolution reduces the computational requirements of the entire network and improves the detection accuracy of the entire model. Compared with the mainstream convolutional neural network, our model is seemed to be more accurate and better in real-time test, which meet the requirements of surface defect detection.
}

\ekeywords{surface defect; machine Vision; deep learning;  Image Processing; nerual network}

\makeabstract