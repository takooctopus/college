\chapter{深度原理}
\section{深度学习系统的基本组成原理}
\subsection{【子系统1】原理}
\subsubsection{提取单个缺陷}
\subsubsection{颜色特征提取}
\subsubsection{纹理特征提取}
\subsubsection{几何特征提取}
\subsection{【子系统2】原理}

\begin{table}[htbp]
	\caption{用语对照表}\label{table}
	\vspace{0.5em}\centering\wuhao
	\begin{tabular}{ccccc}
		\toprule[1.5pt]
		原文 & 翻译 \\
		\midrule[1pt]
		嗯 & angry \\
		啊 & excited \\
		\bottomrule[1.5pt]
	\end{tabular}
	\vspace{\baselineskip}
\end{table}


\chapter{表面缺陷识别系统的开发}
\section{【平台搭建】子架构技术1}
\section{【架构原理】子架构技术2}
\section{【特征提取算法】子架构技术3}

\begin{itemize}
	\item item1;
	\item item2;
	\item item3;
\end{itemize}


\chapter{架构设计}

\begin{figure}[htbp!]
	\centering
	\includegraphics[width=0.5\textwidth]{figures/demo.jpg}
	\caption{demo的图片}\label{book}
	\vspace{-1em}
\end{figure}	

用以公式
\begin{align}
	f(a,b,c)=abc(100a+10b+c)\label{func}
\end{align}

其中\(a,b,c\in \mathbb{N}^+\)且\(a>b>c\),求函数~\eqref{func}~的值域。

\chapter{实验结果与讨论}
结论相关章节

\section{实验一}

\section{实验二}


\chapter{对更广泛应用的评估}
\section{总结}
\section{后续工作展望}


随着社会发展,有更多的⼯件材料被⽣产出来,对于表⾯的质量要求也逐渐 增⾼,表⾯的质量问题也逐渐被⼈们所关注,但是在⼯件⽣产过程中,收到⽣产 ⼯艺或者机器影响,⼯件上经常会出现凹陷、凸起、划痕等表⾯缺陷问题。因此 逐步形成了表⾯缺陷检测技术。表⾯缺陷检测技术是⾃动化⼯业⽣产中的重要 ⼀环,在整个⽣产环节中,产⽣表⾯缺陷,形成部件缺失或者是部件错位,除了 检测精度以外,在资源有限的⾃动表⾯检测解决⽅案上也需要在易⽤性和处理 时间⽅⾯也同时具有优异的性能。本⽂提出了⼀种基于 CornerNet 的神经⽹络结 构,⽤于检测⼀些常⻅的表⾯缺陷。在数据集的基础上,基于图像的降噪、预处 理和数据增强,增加数据的可⽤性。并使⽤多层的 fire module ⽤以优化⽹络,并 使⽤ Hourglass 模块构建整个⽹络结构,通过深层可分离卷积⽤以降低整个⽹络 的运算需求,也提⾼了整个模型的检测精度。实验表明,与主流的卷积神经⽹络
相⽐,我们的模型精确性和实时性都有保证,能够满⾜表⾯缺陷检测的要求。

With the development of morden society, more workpiece are produced, and the surface quality requirements are increasing. To stand test of time, the quality of surface is gradually concerned by people. However, After production process, workpiece may take damage due to the production process and influence of machine,Such as depressions, bumps, and scratches often occur on the workpiece. Therefore, surface defect detection technology has gradually formed and become an important part of automated industrial production. In the production process, missing parts or parts are dislocated may be the surface defects. Despite detection accuracy, it is also required in automatic surface inspection solutions with limited resources. It also need to have excellent performance in terms of ease of use as well as that in processing time. This paper presents a neural network structure based on CornerNet for detecting some common surface defects. Based on the dagm dataset, noise reduction, pre-processing and image augmentation are applyed to incerase data credibility. The entire network structure is built with the hourglass module, and multi-layer fire modules are included to optimize our network. The deep separable convolution reduces the computational requirements of the entire network and improves the detection accuracy of the entire model. Compared with the mainstream convolutional neural network, our model is seemed to be more accurate and better in real-time test, which meet the requirements of surface defect
detection.

表⾯缺陷;机器视觉;深度学习;图像处理;神经⽹络; surface defect; machine Vision; deep learning; Image Processing; nerual network