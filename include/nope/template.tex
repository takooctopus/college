\chapter{深度原理}
\section{深度学习系统的基本组成原理}
\subsection{【子系统1】原理}
\subsubsection{提取单个缺陷}
\subsubsection{颜色特征提取}
\subsubsection{纹理特征提取}
\subsubsection{几何特征提取}
\subsection{【子系统2】原理}

\begin{table}[htbp]
	\caption{用语对照表}\label{table}
	\vspace{0.5em}\centering\wuhao
	\begin{tabular}{ccccc}
		\toprule[1.5pt]
		原文 & 翻译 \\
		\midrule[1pt]
		嗯 & angry \\
		啊 & excited \\
		\bottomrule[1.5pt]
	\end{tabular}
	\vspace{\baselineskip}
\end{table}


\chapter{表面缺陷识别系统的开发}
\section{【平台搭建】子架构技术1}
\section{【架构原理】子架构技术2}
\section{【特征提取算法】子架构技术3}

\begin{itemize}
	\item item1;
	\item item2;
	\item item3;
\end{itemize}


\chapter{架构设计}

\begin{figure}[htbp!]
	\centering
	\includegraphics[width=0.5\textwidth]{figures/demo.jpg}
	\caption{demo的图片}\label{book}
	\vspace{-1em}
\end{figure}	

用以公式
\begin{align}
	f(a,b,c)=abc(100a+10b+c)\label{func}
\end{align}

其中\(a,b,c\in \mathbb{N}^+\)且\(a>b>c\),求函数~\eqref{func}~的值域。

\chapter{实验结果与讨论}
结论相关章节

\section{实验一}

\section{实验二}


\chapter{对更广泛应用的评估}
\section{总结}
\section{后续工作展望}

