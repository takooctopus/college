\chapter{有代表性的主流的缺陷检测模型} 

% [http://www.cjig.cn/html/jig/2017/12/20171202.html]

\section{模板匹配算法}

模板匹配就是将将模板图像与测试图像进行比较,以此来发现缺陷。要在一幅图像中寻找一个特定的目标,只需要遍历所有的位置,并看是否于模板相似,只要高过设定的阈值,我们就认为我们找到了目标。

平均绝对差算法是最为常用的一种算法,其就是对图像进行遍历,对每一次遍历求出其灰度值之差绝对值之和的平均值,最小且在阈值内的就是我们要找到目标。

\begin{eqnarray}
    D(i,j) = \frac{1}{S_temp} \sum_{s} \sum_{t} \lvert S(i+s-1,j+t-1) - T(s,t) \rvert
\end{eqnarray}

还有几个相似的使用的是并非L1距离,而是L2距离。

归一化相关算法也是使用的灰度,但其使用的是归一化相关性来度量:

\begin{eqnarray}
    Relation(i,j) = \frac{\sum_s \sum_t \lvert S^{i,j}(s,t) -E(S^{i,j}) \rvert \cdot \lvert T(s,t) - E(T) \rvert}{\sqrt{\sum_s \sum_t [ S^{i,j}(s,t) -E(S^{i,j}) ] ^{2}   \cdot \sum_s \sum_t [ T(s,t) - E(T) ]^{2}}} 
\end{eqnarray}

序贯相似性检测算法中获取了两图的去掉均值后的对应位置的绝对值之差,选取随机点,将其绝对误差相加,当最终超过阈值时,转向下一个子图,最终没有超过的就能算适配图。

hadamard变换算法引入了hadamard矩阵,这是一个元素全为1或者-1的hermitian矩阵,对变换后的矩阵求其元素的绝对值之和,最小的是最佳匹配。

预处理也是模板匹配的重要一环,图像的旋转、归边、剪裁和平移都是不可或缺的。因为要用到对象的整体特性,拥有不变性的矩形特征被引入计算。非正交矩主要有几何矩、复数矩、旋转矩等。归一化的中心矩对目标图像平移、尺度变换具有不变性。Hu不变矩,具有平移、旋转和比例不变性。正交矩中连续正交矩有Zernike矩、伪Zernike矩、Legendre矩、正交Fourier-Mellin矩,离散正交矩主要有Chebyshev矩、Krawtchouk矩的运用

Hu矩和Zernike矩在旋转不变性上的应用。

\section{特征提取}

特征提取获取图像特点的一个常用方法,对于像木材、纺织物等,其有相似的花纹图案,其不依赖于颜色或亮度,反映了表面结构组织排列的同质性,其拥有者对于局部区域像素的统合特点的综合反映。又在全局中有重复性。

纹理特征的提取有几种常用方式,最简单的是统计方法,直方图用于统计的灰度特征是最为简单的特征,除开一般的数值比较特征、统计的范数特征、归一化相关系数都能运用。

灰度共生矩能够反映图像灰度的方向、间隔、变化信息,其统计了图像 $(x,y)$ 处的灰度级为 $i$ 的像素与周围距离一定内,灰度级为j像素同时出现的概率。灰度共生矩的主要特征就是对比度、相关性和能量。

局部二值模式将各个像素与其附近的像素进行比较,并将结果保存为二进制数。

相关函数的运用也是能被考虑的,对于规则纹理图像,自相关函数能显示出波峰和波谷,图像能量谱函数的子相关函数可以由此提取纹理方向和大小参数。

提到特征,除了能量领域的特征,频域上的特征也是可以分离出来的,这时我们就需要频域滤波器。

首选的是傅里叶变换,功率谱已经能够一定程度上获取图像的空间分布特征了,但方向性在频率谱上保存很好。特别是对于纹理粗糙程度,频率谱上对于频率中心的距离远近能反映纹理的粗糙程度,越近就代表越为细致。

傅立叶变换是全局变换,要获得区域性的纹理特征,Gabor滤波器能够很好的提取纹理特征。Gabor滤波器是一种特殊的短时窗口傅里叶变换(STFT),它在变换时增加一个高斯窗函数来实现,模拟了人类视觉感觉特,有很好的频率选择性和方位选择性。

与此同时,小波变换也被引入到了图像处理中,其通过伸缩和平移等运算功能对函数和信号进行细化。

\section{缺陷识别}

缺陷识别算法分为监督学习和无监督学习。

无监督学习主要是以聚类分组揭示模式,将对象分为不同的组,我们的聚类法有很多,划分法、层次法、密度法、网格法、图论、模型法都是较为常见的聚类方法。此外针对数据不明确的模糊聚类、模拟退火聚类、迭代自组织数据分析算法和粗糙集方法以及基于遗传学的聚类、蚁群聚类算法粒、子群聚类算法也被用到图像识别和缺陷分析中。

监督学习就是在已知类别标签的特征集基础上进行分类器构建,一般的图像数据都需要训练数据的支撑,最简单的是基于概率论的分类器,贝叶斯理论下的简易贝叶斯分类器适用于有统计知识且能利用训练样品估计出参数的场合。

人工神经网络是监督学习中的一个大子类,具有非线性、自适应、自组织、自学习能力、非局限性、非凸性和容错性等一系列特点。在这上面的运用比如说对抗性样本和自动编码器生成,先结合GAN和Autoencoder算法进行图像样本修复,再利用LBP算法比较恢复后的图像和原始输入图像,从而更准确的找到缺陷的位置。或是卷积神经网络的实现,通过卷积神经网络下的池化操作和稀疏连接获取图像总体特征,能够获得超越传统先验知识下的手工分类器,如先验阈值法等。针对CNN的缺陷,还有引入全卷积网络和残差网络的想法。深度学习网络分为one-stage和two-stage两种,分别是直接在图片上经过计算生成detections和先提取proposal, 再基于proposal做二次修正。此外,基于heatmap的算法也能够在两个方向上获得不同的特征。人工神经网络做为整个方法的一部分,与其他技术的结合取长补短,能形成很多混合方法和混合系统。

\section{本章总结}
本章叙述了三个主流的缺陷检测模型,从不同的方向上进行了讨论。基于图像处理的简单木板匹配算法是最为基础的算法,但我们在处理时仍旧需要考虑到模板和图像的旋转分割等问题。特征提取也是我们的一个大方向,特别对于纹理特征,空间域和频域特征同样重要。最后我们探讨了机器学习算法中无监督学习和监督学习算法的不同以及他们各自的应用优势。