\chapter{绪论}
\section{研究背景和意义}

缺陷是传统工业制造中不可避免的部分,由于现有生产条件、科学技术的限制,缺陷广泛的出现在产品之中,为了产品控,我们也需要引入缺陷检测技术。

但传统的缺陷检测主要由人工检测,采用大量人力,但由于技术落后、判定不清晰、人工的限制,检测时的速度无法保证、效率也很难提升,错误率居高不下,无法达到期待的检测结果。

在近些年,芯片设计、软件设计的发展、人工智能和图像识别的再兴起,出现了基于机器视觉的表面缺陷检测技术和基于机器学习的表面缺陷检测技术。由此能够避免由于人主观判断而影响检测结果的准确,大大提高进行缺陷检测和分类的准确性和时效性。

在过去的时间中,表面缺陷检测中的机器视觉技术也被广泛的进行了研究。基于机器视觉的检测系统已经出现在很多地方,像是钢材%[[2]]
、液晶屏幕%[[4]]
、针织产品%[[6]]
、铁路%[[8]]
、食品%[[9]]
、水果%[[N]]
还有光学器件%[[10]]
中。

我们进行的基于卷积神经网络的框架,将基于角网络的卷积深度神经网络和残差网络相结合,分路采用降分辨采样和上采样进行不同处理,最后合并后能够有效提高检测效率和准确率。

\section{国内外发展现状}

% [[Automatic Metallic Surface Defect Detection and Recognition with Convolutional Neural Networks]]

% [https://sci-hub.tw/https://www.sciencedirect.com/science/article/pii/S1474034615000208]

% 「https://sci-hub.tw/https://www.sciencedirect.com/science/article/pii/S0030402616311366」

% (https://sci-hub.tw/https://onlinelibrary.wiley.com/doi/abs/10.1111/mice.12263)

% {A Contrast Adjustment Thresholding Method for Surface Defect Detection Based on Mesoscopy}

表面缺陷检测最初始的就是缺陷的定义,比如在桥梁元件检验手册中,就规定了必须记录的钢筋混泥土的表面缺陷:分层/剥落/修补区域,钢筋外露,风化/锈蚀,开裂,磨损/磨损,变形,沉降和冲刷等,将其列出了完整的缺陷清单,除开这些,还有关于轴承的相关缺陷,可能是连接问题、过度运动、器件膨胀和撕裂、密封性损坏、开裂、碎片或者金属劣化。在小型器件上也可能会出现这些问题,但在刚出厂的器件上一般的风化、腐蚀等问题并不常见。

通常情况下,上面的问题都可以使用目视来检查,最简单的人工检查需要一个检查表,用以记录可以检测的缺陷。对于有些缺陷并非能够完全由目测检测到,或者说是由目测能够定义,这时需要在物理上进行深入检查,目测并无法深入了解到缺陷的深度。为此先进的基于物理的声波、超声波、磁、电、核、热成像、雷达等技术都能被运用到表面缺陷检测上去。\cite{doi:101061}

标准对照表的使用是在人工进行表面缺陷检测的重要工具,标准表被工程师用于评估缺陷的成分和类型。因此,在很多领域我们可以看到标准手册的身影,比方说对于土木行业的运营、维护、检查和评估手册和各种统一准则。这些的运用可以整合结构表面缺陷的位置、类型、大小和数量,有些使用0-9的评级系统,将0视做最坏情况。\cite{koch2015review}

缺陷检查也并非只有一次,对于消费领域的比如说电子元器件,缺陷检测可能只需要在出厂的时候检测合格即可,最多留下部分期间做寿命测试后再进行二次检测。但对于铁轨、桥梁、道路等公众设施,设置合适的检测周期也是必不可少的。一般的、除了初始检查,对于长期使用的设施,例行检查、损坏检查、深入和特殊检查\cite{operations2015evaluation}也是必要的。

\subsection{图像预处理的研究}

不同于一般在工厂中,可能不受天气和光照的影响,有相对稳定的环境,很多情况下图片的获取在室外,有不同的角度和光线条件,或者阴影的干扰,这种阴影错觉,可能两个表面缺陷亮度看起来并不相同,由于其背景的强度的不同,人可能会得出不同的结论,但是计算机更多的是依赖于全局强度值。

一般的,我们会选择一个标准的拍摄条件,这能够帮助我们省去不少图像处理的工作,比如在\cite{varadharajan2014vision}中就只选择多云天的白天拍摄,以保证光照良好并且不会有太强烈的阳光。如果最开始并没有那么好的拍摄条件,我们可以使用直方图平均全局光照,或是去区域标准化光照,或是将整个图片拆分成数个矩形中,并在其中计算其平均强度,再使用池化将本区域的值替换为周围矩形区域的平均值。\cite{ZOU2012227}中提出了一个阴影的去除算法。\cite{nguyen2009automatic}对图像先进行阈值处理,再进行Hough变换。

除开环境的影响,相机的选择也是一个很重要的因素,不同的相机和镜头能够产生不同的图像,为之后的图像处理提供方便。比方说区域扫描相机就更适合于低速的物体,运动图像会显得模糊,如果是流水线上的缺陷检测,很可能处于运动状态,因此,对于不停移动的物体表面,可以选择线扫描相机,graftek公司就提出了一种线扫描相机。镜头决定了查看区域和视野,最终的图像形状和大小与镜头也息息相关。\cite{Buck2013}中的光学显微镜用于表面检测以捕获高分辨率图像,尽管它具有有限的视野并且更适合于小样本,最终用于2.75um每像素的钢表面缺陷检测上。

图像分析中很重要的一个步骤就是阈值处理,其利用强度转换来生成灰度或彩色图像的黑白二值图像,\cite{Kittler:1986:MET:2035720363}中提到了一种最小误差阈值方法。\cite{win2015contrast}中考虑了图像的对比度和中值,提出了利用对比度调整的Otsu方法和对比度调整中位数Otsu方法。

图像增强就是一种能够很好地突出图像的可辨识度的策略。其能够突出图像中的有用信息,削弱图像中的无用信息。而图像增强算法也在空间域、频域、对比度、照度方面分别有其各自的着重点。我们分别有直方图均衡算法,用以增强整个图像的对比度,使得其更为平坦也更为均匀。小波也是运用的方法之一,小波变换将离散二维信号图像分成了1个低通子图像和3个高通子图像。分别对应于图像中平滑区域和细节区域。而高通图像分别包括了水平、垂直和对角的细节图像。此外我们还有偏微分方程图像增强和Retinex图像增强。

\subsection{基于计算机视觉的表面缺陷检测研究现状}

[图1]

\cite{doi101080}中对以前的自动缺陷检测方法进行了回顾。对于混凝土缺陷检测和评估,\cite{rose2014supervised}中将其分成了边缘检测、分割渗透、形态学、模板匹配和其他技术。\cite{7864335}中还提到了结构法和形态学的运用。

边缘检测是一个方面,\cite{doi101061}中比较了边缘检测算法的优劣,并着重强调了Haar小波的意义。边缘检测上噪声是不可忽视的一点。\cite{Yamaguchi2010}中使用的可拓展的局部渗漏图像处理技术,能对大型的表面图像也能有优秀的性能。\cite{abdel2006pca}中使用了主成分分析,准确性随相机位置而有所变化,基于直方图分类和支持向量机的方法,使用不同位置数据构建了简易分类器。\cite{doi0000257}中使用了一种基于分割的自动聚类方法,能有效缓解由于光照条件变化而引起的损失。

基于阈值的方法也是一个可行的方法,计算每个像素和周围像素平均值的差异,过大则将其看作一个缺陷,或是使用了基于Hessian矩阵的线性滤波器用以从背景中分离缺陷。但由于我们缺陷检测中缺少相应的先验知识,很难提前选择阈值,在工厂条件下由于光超和测试条件可控,阈值的应用和控制较为简易,但是在现实环境中,不太适用于真实的成像条件,正如光照和角度的影响,阴影是很重要的影响因素。\cite{doi:1010610000051}中就除了统计阈值外,还使用了Canny边缘检测、多尺度小波、缺陷种子验证,迭代剪切方法和基于动态优化的方法。

基于模型的检测方法也是有所发展,\cite{Yamaguchi2010}中使用的类似于液体渗透的概念,通过种子区域来标注相邻区域用作为渗漏过程区域。或是线追踪算法,通过种子开始搜索周围区域的线,将整个缺陷视作一起的短直线。模型的使用强烈的依赖着种子的选取,种子像素的选取可能是随机的,有时也有相应的侧重,但对于很细的缺陷,种子选取很重要。种子位置的选择很可能决定这个缺陷会不会被检测到。\cite{Masato2000}中提到了高斯马尔可夫随机场模型。又或是使用自回归模型表示了图像的不同像素中的相关性。\cite{SUSAN2017232}中提及高斯混合熵模型,\cite{CEN20151206}中讨论了低秩矩阵模型的应用。

特征提取也是计算机视觉的重要应用,支持向量机和最近邻分类器,给出一个最后要分类的对象的值向量。特征的提取就是降维,边缘检测、Hough变换中就是边缘检测和线检测。\cite{SONG2013858}是局部二值模式特征,\cite{doi:10106108873801}中应用了前景分割,为了提取物体,\cite{Guo2009}中的背景减法也是一种参考方法。基于小波变换的方法可以适用于分辨率较高的图像,而纹理方法也是一个方向,共生直方图、梯度直方图,将缺陷建模为一组比周围区域更暗的区段。 \cite{647318}中使用空间灰度级依赖矩阵来改进你对纹理缺陷的描述。\cite{LATIFAMET2000543}中使用了带灰度的共生矩阵的方法研究了织物缺陷灰度共生矩阵也在\cite{Chondronasios2016}中有所阐述。\cite{BISSI2013838}中就使用了住成分分析用以进行缺陷检测。

形态学也是一个研究方向,基本图像处理中的膨胀,侵蚀,开启和关闭是一个常见的操作,通过控制图形的侵蚀和生长,能够以此分析一个或多个圆形或矩形结构组成出的图案。\cite{Shahin1997}中就给出了一种形态学侵蚀算子。\cite{SINHA200647}中有在管道中对形态学的应用。\cite{Moussa2011}中带入了图像分割,将图像分为了裂缝和背景像素,并由此提取了7个主特征,用支持向量机对不同方向的裂缝类型进行了分类。或是使用椭圆来近似坑洞的表现。\cite{Radopoulou2014}中假设了贴片像素具有比背景更大的强度,使用了纹理信息并进行了滤波处理。对于有些有纹理的器件,可以使用结构处理,其将表面纹理视作纹理原语的组合,根据其对纹理的特征进行推断。\cite{ABOUELELA20051435}中就使用了类似的结构缺陷检测。\cite{6287161}中提到了形态学方法和小波变换的组合使用。\cite{7334956}中提及了最优Gabor滤波器的标准。\cite{Halimi2012}也用到了形态学技术和几何形状数据。(2003)中就还讨论了快速Haar变换、快速傅里叶变换、Sobel边缘检测器和Canny边缘检测器的优劣。

基于光谱的方法也是被计入讨论的,特别是频域信息必不可少,特别对于有纹理的表面,纹理的存在有高度的周期性,频域上对于周期性信息有很好的分辨效果。\cite{TABASSIAN20115259}中提出了基于小波变换的方法,以3个小波去主要表示缺陷。\cite{6720367}中研究了纹理表面缺陷的小波变换方法。傅立叶变换也能进行缺陷检测,其在高速流水线上具有优势。Gabor变换也是一大方向,Gabor滤波器用于在空间和频域进行纹理分析,这些滤波器可以定制,其可以根据纹理结构具有不同的比例和角度值。

3D场景重建是算是结果的聚合,将整个缺陷映射到全局坐标系,或是拼接了多个图像,获取结构姿势。\cite{Jahanshahi2013}提出了一个缺陷检测系统来提取完整​​的缺陷地图使用3D场景重建,形态学操作和机器学习分类器。3D重建能够直接获取到缺陷和孔洞。立体视觉中很重要的就是阴影和形状。为此激光重建的使用很普遍,\cite{1232300}中就描述了一种三维重建的方法。\cite{Li2009}中引入了一种基于立体视觉的路面调节方法。\cite{Karuppuswamy2000}使用了计算机模拟坑洞的形成。\cite{doi:10106119435487}中使用了微软kinect传感器来尝试在室外测量深度。

\subsection{基于深度学习的表面缺陷检测研究现状}


传统方法基于图像处理或浅机器学习技术,但这些方法只能检测特定检测条件下的缺陷,例如在特定照明条件下具有强对比度和低噪声的明显缺陷轮廓。其对照明条件和背景颜色非常敏感,因为需要设置对于缺陷的多个阈值。如果出现新的问题,则我们需要对其进行调整,甚至于调整整个算法,这种明显的适应性和鲁棒性缺损是手工或先验特征的缺陷。

但现在有很多类型的人工神经网络被运用到到了缺陷检测的识别上面,能够有效的针对于现实世界的光照和阴影情况作出很好的效果。与此同时最开始的就是概率神经网络,\cite{6706920}中提到了一种多尺度的金字塔型结构。

卷积神经网络是受到了动物大脑皮层在视觉反映上的启发,整个网络结构和标准NN网络不同,其并没有完全使用全连接层,而是采用了更为稀疏的神经元结构并加入了池化过程,(LeCun等,2015)就表示了这个是更为有效图像识别方法。其多层结构能自动学习特征,并且可以学习到多个层次的特征。CNN需要很多标记好的数据,如用于手写识别的MNIST数据库,或是IMAGENET上的数据库,大型的分类好的带标签的数据集是必要的,而这也会带来高昂的运算代价,为此需要大型的图形处理单元用于计算。数据集的生成也可以自动产生,但是要控制原始图片的生成条件,如Soukup就讨论了铁轨的图像收集问题,由于铁轨表面应该是均匀且光滑的,在受控条件下采集的图像就可以自动进行分类检测。\cite{Wang2018}中使用了传统CNN与滑动窗口的结合用以定位缺陷位置。\cite{Lin2018}中也使用了CNN来识别LED的缺陷。

在传统卷积神经网络的基础上,将最后的全连接层换为卷积层,从抽象的特征中恢复出每个像素所属的类别。即从图像级别的分类进一步延伸到像素级别的分类。在最后更改后CNN的特征图由于卷积层的设置,最后的卷积层出来的图我们称其为热度图。最后使用上采样并逐级相加,就能够得到整个图像的特征。在\cite{2018SPIE10615E0KX}中就使用了全卷积网络检查镀锌冲压件。

另一个方向就是卷及网络的深度,一般的深度的增加能有效增加最终网络的判断准确度,但是会大幅增加所需的资源和时间。\cite{8126877}中就存在一个使用了3个深层卷积神经网络的诊断系统,分别用于定位和状态判断。\cite{7864335}中使用了预训练的网络,通过热度图分割对缺陷图像进行分割。\cite{app8091575}中使用了一种级联自动编码器,将整个缺陷检测和识别任务表示为分段和分类问题,可以获得更准确和一致的缺陷检测结果。

近年来的图像识别方向上有一些新的想法,我们最终需要将缺陷图像表示出来,为此,基于border的学习算法是很值得期待的。一般的研究大多将其放在了图像识别上,除了\cite{NIPS2012_4824}提到的ALex网络,\cite{Simonyan15}中的VGG在其上进行了改进,使用了3个3x3卷积核来代替7x7卷积核,使用了2个3x3卷积核来代替5*5卷积核,验证了通过不断加深网络结构可以提升性能。\cite{Redmon2018YOLOv3AI}的YOLOv3使用了全新设计的Darknet53残差网络并结合FPN网络结构。SSD中前后的卷积层分别检测不同大小的目标。\cite{duan2019centernet}中提出了基于中心点的CenterNet网络用以目标检测,其使用左上和右下两个角点生成初始目标框,对每个预测框定义一个中心区域,判断每个目标框的中心区域是否含有中心点来决定况的保留与否。\cite{DBLP:journals/corr/abs-1808-01244}中提出了基于左上角和右下角点的border识别模型。并在\cite{DBLP:journals/corr/abs-1904-08900}中提出了改进型CornerNet-Lite引入了人眼扫视原理,提升了准确度,并由SqueezeNet\cite{DBLP:journals/corr/IandolaMAHDK16}和MobileNet\cite{DBLP:journals/corr/HowardZCKWWAA17}启发,提升了残差结构的效率。目标识别技术的发展也是我们很重要的参考方向。

\section{本论文的结构安排}
本论文设定为六章,在第一章中我们对论文的研究背景和意义进行了梳理,叙述了进行表面缺陷识别工作研究的必要性,在这一章中,我们对近年来的表面缺陷识别工作的发展,归纳了现在的主流表面缺陷识别的各种方法,从预处理、图像特征和神经网络的不同方面进行了阐述。第二章我们对主流的表面缺陷识别框架进行了叙述,三个主流方向模板匹配、特征提取和缺陷识别,分别简述了其原理和发展。第三章我们回顾深度学习的基本理论知识,卷积、池化、激活、层间设置、损失函数、优化方法、改进方法都均有叙述。第四章是我们的网络结构,这一章中,我们详细叙述了我们使用的CornerNet-lite的网络结构原理,以及我们在此基础上实现的改进。数据集增强也运用到我们的数据集中,以改善我们训练的样本数量。在第五章中写出了我们的实验分析和我们对其优化的方式,以及与其他的实验的对比判断。在最后一章中我们总结了我们实验方法的不足,提出了一些可能的修改方向。

