\chapter{总结和展望}
\section{论文总结}
本文实现了基于Hourglass结构的深层卷积神经网络结构实现的表面缺陷检测识别系统,通过Corner pool和Hourglass module的结合,我们能够有效的提高检测精度。并使用了fire module 和 深层可分离卷积技术,尽量的在保证准确性的情况下减少我们的运算量,提升了我们的效率。

我们在第一章的时候回顾了过去这些年间在表面缺陷识别领域的研究和成果,论述了表面缺陷领域的各个不同的方法的异同。对表面缺陷识别问题下的预处理、边缘检测、阈值分析、模型检测和特征提取等方面进行了简单的介绍,分析了其优缺点,并了解了缺陷检测的发展历程和目前问题。

在第二章我们回顾了主流的表面缺陷检测模型,从图像处理的模板匹配算法开始,与图像的旋转等性质结合,分析表面图像的特征提取方法。统计、灰度共生矩、相关函数、频域分析、小波变换都被引入讨论。我们还简单探讨了基于机器学习的缺陷检测算法,分析了无监督学习和监督学习在目标识别问题上的异同和优劣。

第三章我们简述了在深度学习的目标识别问题上的基本原理,叙述了在深度学习算法中的卷积、采样、池化、分类、反向传播和优化等原理,简述了我们现在采用的理论和工具的优势,简述了我们缺陷目标检测的步骤和原理。

在第四章中我们提出了我们缺陷检测系统的组成实现。研究了缺陷图像的特征,探讨了缺陷识别问题的目标,考虑了效率和精度上的要求,探究了缺陷识别系统的可能性。我们详细地列出了基于hourglass模块的精度检测系统的具体网络结构和实现,并研究了为何设置角池化和如何设置损失函数以达到优化的目标。叙述了为何理论上我们的网络有相应的精度和效率。

第五章我们对我们的实验结果进行了分析和总结,我们和以往的基于本数据集的研究结果进行对比,比较我们的检测精度,在另一边,我们也同时记录了网络运行时间,比较了其和之前各网络的效率。

\section{未来展望}

本文使用的基于CornerNet的缺陷检测系统在基于dagm2007数据集下实现了很好的检测效果,但仍旧有一些问题亟需完善:

\begin{enumerate}
    \item 本文使用的是dagm数据集只有10类数据,共8000张图像,其中还有大量没有标注和缺陷的无效样本,在本数据集中的表现结果并不一定能够表征在真实情况下的缺陷检测情况。
    \item 本方法使用的缺陷检测是基于图像边框的对点的检测,十分基于我们预设的检测边框,但对于很多缺陷来说,比如划痕、凹陷,其呈长条型,边框会包括过多无关区域,只能获取缺陷的整体位置。
    \item 本文的输入数据需要特殊的转换格式,并且对于图像大小和质量有一定的要求。我们需要改进其成为一个更为通用的网络结构。
    \item 本文的网络结构虽然有输入要求,但是我们可以进行拓展,使用接口提供一个更为通用的功能耦合,用以解决研究更为鲁棒的目标识别问题。
    \item 目标识别的技术发展迅速,每年都会出现新的具有突破性的网络结构,训练工具性能的增加也会显著的提升我们训练网络的精度和速度,能够提供条件实现更深的网络结构。
\end{enumerate}
