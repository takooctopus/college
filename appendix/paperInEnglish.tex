% !Mode:: "TeX:UTF-8"

\titlecontents{chapter}[2em]{\vspace{.5\baselineskip}\xiaosan\song}%
             {\prechaptername\CJKnumber{\thecontentslabel}\postchaptername\qquad}{} %
             {}             % 设置该选项为空是为了不让目录中显示页码          
\addcontentsline{toc}{chapter}{外文资料}
%\setcounter{page}{1}       % 如果需要从该页开始从 1 开始编页,则取消该注释
\markboth{外文资料}{外文资料}
\chapter*{外文资料}

\trtitle{Deep Learning based Recommender System: A Survey and New Perspectives}

Œe explosive growth of information available online frequently overwhelms users. Recommender system (RS) is a useful information €ltering tool for guiding users in a personalized way of discovering products or services they might be interested in from a large space of possible options. Recommender system has been playing a more vital and essential role in various information access systems to boost business and facilitate decision-making process.

In general, the recommendation lists are generated based on user preferences, item features, user-item past interactions and some other additional information such as temporal and spatial data. Recommendation models are mainly categorized into collaborative €ltering, content-based recommender system and hybrid recommender system based on the types of input data . However, these models have their own limitations in dealing with data sparsity and cold-start problems, as well as balancing the recommendation qualities in terms of di‚erent evaluation metrics